\documentclass[12pt,a4paper]{report}


\usepackage{siunitx}

\usepackage{listings}
\usepackage[utf8]{inputenc}
\usepackage[polish]{babel}
\usepackage{lmodern}
\usepackage[T1]{fontenc}
\usepackage[babel=true]{microtype}

\usepackage[hidelinks]{hyperref}

\author{my!}
\title{3 Słowa o git}

\lstset{
  frame=single,
  captionpos=b,
  basicstyle=\scriptsize,
  postbreak=\raisebox{0ex}[0ex][0ex]{\ensuremath{\hookrightarrow\space}},
  breaklines=true
}

\begin{document}
\maketitle

\chapter{Inicjalizacja repo}
Aby sklonować repo u siebie na dysku należy wykonać polecenie:
\begin{lstlisting}[language=bash]
  $ git clone ADRES REPO
\end{lstlisting}
np:
\begin{lstlisting}[language=bash]
  $ git clone git@github.com:WMP/GitLab2.git
\end{lstlisting}


Po zainicjalizowaniu repo nalezy dodać upstream do głównego repo, aby móc pobierac zmiany z repo, które forkowaliśmy:
\begin{lstlisting}[language=bash]
  $ git remote add upstream https://github.com/makas9393/lab.git
\end{lstlisting}

Aby odświeżyć zmiany należy wykonać:
\begin{lstlisting}[language=bash]
  $ git fetch
  $ git rebase origin/master
\end{lstlisting}

\end{document}
